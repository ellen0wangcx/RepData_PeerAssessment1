\documentclass[]{article}
\usepackage{lmodern}
\usepackage{amssymb,amsmath}
\usepackage{ifxetex,ifluatex}
\usepackage{fixltx2e} % provides \textsubscript
\ifnum 0\ifxetex 1\fi\ifluatex 1\fi=0 % if pdftex
  \usepackage[T1]{fontenc}
  \usepackage[utf8]{inputenc}
\else % if luatex or xelatex
  \ifxetex
    \usepackage{mathspec}
  \else
    \usepackage{fontspec}
  \fi
  \defaultfontfeatures{Ligatures=TeX,Scale=MatchLowercase}
\fi
% use upquote if available, for straight quotes in verbatim environments
\IfFileExists{upquote.sty}{\usepackage{upquote}}{}
% use microtype if available
\IfFileExists{microtype.sty}{%
\usepackage{microtype}
\UseMicrotypeSet[protrusion]{basicmath} % disable protrusion for tt fonts
}{}
\usepackage[margin=1in]{geometry}
\usepackage{hyperref}
\hypersetup{unicode=true,
            pdftitle={PeerAssessment1},
            pdfborder={0 0 0},
            breaklinks=true}
\urlstyle{same}  % don't use monospace font for urls
\usepackage{color}
\usepackage{fancyvrb}
\newcommand{\VerbBar}{|}
\newcommand{\VERB}{\Verb[commandchars=\\\{\}]}
\DefineVerbatimEnvironment{Highlighting}{Verbatim}{commandchars=\\\{\}}
% Add ',fontsize=\small' for more characters per line
\usepackage{framed}
\definecolor{shadecolor}{RGB}{248,248,248}
\newenvironment{Shaded}{\begin{snugshade}}{\end{snugshade}}
\newcommand{\KeywordTok}[1]{\textcolor[rgb]{0.13,0.29,0.53}{\textbf{#1}}}
\newcommand{\DataTypeTok}[1]{\textcolor[rgb]{0.13,0.29,0.53}{#1}}
\newcommand{\DecValTok}[1]{\textcolor[rgb]{0.00,0.00,0.81}{#1}}
\newcommand{\BaseNTok}[1]{\textcolor[rgb]{0.00,0.00,0.81}{#1}}
\newcommand{\FloatTok}[1]{\textcolor[rgb]{0.00,0.00,0.81}{#1}}
\newcommand{\ConstantTok}[1]{\textcolor[rgb]{0.00,0.00,0.00}{#1}}
\newcommand{\CharTok}[1]{\textcolor[rgb]{0.31,0.60,0.02}{#1}}
\newcommand{\SpecialCharTok}[1]{\textcolor[rgb]{0.00,0.00,0.00}{#1}}
\newcommand{\StringTok}[1]{\textcolor[rgb]{0.31,0.60,0.02}{#1}}
\newcommand{\VerbatimStringTok}[1]{\textcolor[rgb]{0.31,0.60,0.02}{#1}}
\newcommand{\SpecialStringTok}[1]{\textcolor[rgb]{0.31,0.60,0.02}{#1}}
\newcommand{\ImportTok}[1]{#1}
\newcommand{\CommentTok}[1]{\textcolor[rgb]{0.56,0.35,0.01}{\textit{#1}}}
\newcommand{\DocumentationTok}[1]{\textcolor[rgb]{0.56,0.35,0.01}{\textbf{\textit{#1}}}}
\newcommand{\AnnotationTok}[1]{\textcolor[rgb]{0.56,0.35,0.01}{\textbf{\textit{#1}}}}
\newcommand{\CommentVarTok}[1]{\textcolor[rgb]{0.56,0.35,0.01}{\textbf{\textit{#1}}}}
\newcommand{\OtherTok}[1]{\textcolor[rgb]{0.56,0.35,0.01}{#1}}
\newcommand{\FunctionTok}[1]{\textcolor[rgb]{0.00,0.00,0.00}{#1}}
\newcommand{\VariableTok}[1]{\textcolor[rgb]{0.00,0.00,0.00}{#1}}
\newcommand{\ControlFlowTok}[1]{\textcolor[rgb]{0.13,0.29,0.53}{\textbf{#1}}}
\newcommand{\OperatorTok}[1]{\textcolor[rgb]{0.81,0.36,0.00}{\textbf{#1}}}
\newcommand{\BuiltInTok}[1]{#1}
\newcommand{\ExtensionTok}[1]{#1}
\newcommand{\PreprocessorTok}[1]{\textcolor[rgb]{0.56,0.35,0.01}{\textit{#1}}}
\newcommand{\AttributeTok}[1]{\textcolor[rgb]{0.77,0.63,0.00}{#1}}
\newcommand{\RegionMarkerTok}[1]{#1}
\newcommand{\InformationTok}[1]{\textcolor[rgb]{0.56,0.35,0.01}{\textbf{\textit{#1}}}}
\newcommand{\WarningTok}[1]{\textcolor[rgb]{0.56,0.35,0.01}{\textbf{\textit{#1}}}}
\newcommand{\AlertTok}[1]{\textcolor[rgb]{0.94,0.16,0.16}{#1}}
\newcommand{\ErrorTok}[1]{\textcolor[rgb]{0.64,0.00,0.00}{\textbf{#1}}}
\newcommand{\NormalTok}[1]{#1}
\usepackage{graphicx,grffile}
\makeatletter
\def\maxwidth{\ifdim\Gin@nat@width>\linewidth\linewidth\else\Gin@nat@width\fi}
\def\maxheight{\ifdim\Gin@nat@height>\textheight\textheight\else\Gin@nat@height\fi}
\makeatother
% Scale images if necessary, so that they will not overflow the page
% margins by default, and it is still possible to overwrite the defaults
% using explicit options in \includegraphics[width, height, ...]{}
\setkeys{Gin}{width=\maxwidth,height=\maxheight,keepaspectratio}
\IfFileExists{parskip.sty}{%
\usepackage{parskip}
}{% else
\setlength{\parindent}{0pt}
\setlength{\parskip}{6pt plus 2pt minus 1pt}
}
\setlength{\emergencystretch}{3em}  % prevent overfull lines
\providecommand{\tightlist}{%
  \setlength{\itemsep}{0pt}\setlength{\parskip}{0pt}}
\setcounter{secnumdepth}{0}
% Redefines (sub)paragraphs to behave more like sections
\ifx\paragraph\undefined\else
\let\oldparagraph\paragraph
\renewcommand{\paragraph}[1]{\oldparagraph{#1}\mbox{}}
\fi
\ifx\subparagraph\undefined\else
\let\oldsubparagraph\subparagraph
\renewcommand{\subparagraph}[1]{\oldsubparagraph{#1}\mbox{}}
\fi

%%% Use protect on footnotes to avoid problems with footnotes in titles
\let\rmarkdownfootnote\footnote%
\def\footnote{\protect\rmarkdownfootnote}

%%% Change title format to be more compact
\usepackage{titling}

% Create subtitle command for use in maketitle
\newcommand{\subtitle}[1]{
  \posttitle{
    \begin{center}\large#1\end{center}
    }
}

\setlength{\droptitle}{-2em}

  \title{PeerAssessment1}
    \pretitle{\vspace{\droptitle}\centering\huge}
  \posttitle{\par}
    \author{}
    \preauthor{}\postauthor{}
    \date{}
    \predate{}\postdate{}
  

\begin{document}
\maketitle

\subsection{Loading the data}\label{loading-the-data}

\begin{itemize}
\tightlist
\item
  Load the data
\end{itemize}

\begin{Shaded}
\begin{Highlighting}[]
\NormalTok{activity <-}\StringTok{ }\KeywordTok{read.csv}\NormalTok{(}\StringTok{"activity.csv"}\NormalTok{)}
\end{Highlighting}
\end{Shaded}

\begin{itemize}
\tightlist
\item
  Process/transform the data(if necessary) into a format suitable for
  your analysis
\end{itemize}

\begin{Shaded}
\begin{Highlighting}[]
\NormalTok{totalSteps<-}\KeywordTok{aggregate}\NormalTok{(steps}\OperatorTok{~}\NormalTok{date,}\DataTypeTok{data=}\NormalTok{activity,sum,}\DataTypeTok{na.rm=}\OtherTok{TRUE}\NormalTok{)}
\end{Highlighting}
\end{Shaded}

\subsection{The mean total number of steps took per
day.}\label{the-mean-total-number-of-steps-took-per-day.}

\begin{itemize}
\tightlist
\item
  Make a histogram of the total number of steps taken each day
\end{itemize}

\begin{Shaded}
\begin{Highlighting}[]
\KeywordTok{hist}\NormalTok{(totalSteps}\OperatorTok{$}\NormalTok{steps)}
\end{Highlighting}
\end{Shaded}

\includegraphics{PeerAssessment1_files/figure-latex/unnamed-chunk-3-1.pdf}

\begin{itemize}
\tightlist
\item
  Calculate and report the \textbf{mean} and \textbf{median} total
  number of steps taken per day
\end{itemize}

\begin{Shaded}
\begin{Highlighting}[]
\KeywordTok{mean}\NormalTok{(totalSteps}\OperatorTok{$}\NormalTok{steps)}
\end{Highlighting}
\end{Shaded}

\begin{verbatim}
## [1] 10766.19
\end{verbatim}

\begin{Shaded}
\begin{Highlighting}[]
\KeywordTok{median}\NormalTok{(totalSteps}\OperatorTok{$}\NormalTok{steps)}
\end{Highlighting}
\end{Shaded}

\begin{verbatim}
## [1] 10765
\end{verbatim}

\begin{itemize}
\item
  The \textbf{mean} total number of steps taken per day is
  1.0766189\times 10\^{}\{4\} steps.
\item
  The \textbf{median} total number of steps taken per day is 10765
  steps.
\end{itemize}

\subsection{The average daily activity
pattern}\label{the-average-daily-activity-pattern}

\begin{itemize}
\tightlist
\item
  Make a time series plot (i.e.~type = ``l'') of the 5-minute interval
  (x-axis) and the average number of steps taken, averaged across all
  days (y-axis)
\end{itemize}

\begin{Shaded}
\begin{Highlighting}[]
\NormalTok{stepsInterval <-}\StringTok{ }\KeywordTok{aggregate}\NormalTok{(steps }\OperatorTok{~}\StringTok{ }\NormalTok{interval, }\DataTypeTok{data =}\NormalTok{ activity, mean, }\DataTypeTok{na.rm=}\OtherTok{TRUE}\NormalTok{)}
\KeywordTok{plot}\NormalTok{(steps }\OperatorTok{~}\StringTok{ }\NormalTok{interval, }\DataTypeTok{data =}\NormalTok{ stepsInterval, }\DataTypeTok{type=}\StringTok{"l"}\NormalTok{)}
\end{Highlighting}
\end{Shaded}

\includegraphics{PeerAssessment1_files/figure-latex/unnamed-chunk-5-1.pdf}

\begin{itemize}
\tightlist
\item
  Which 5-minute interval, on average across all the days in the
  dataset, contains the maximum number of steps?
\end{itemize}

\begin{Shaded}
\begin{Highlighting}[]
\NormalTok{stepsInterval[}\KeywordTok{which.max}\NormalTok{(stepsInterval}\OperatorTok{$}\NormalTok{steps),]}\OperatorTok{$}\NormalTok{interval}
\end{Highlighting}
\end{Shaded}

\begin{verbatim}
## [1] 835
\end{verbatim}

\begin{itemize}
\tightlist
\item
  Devise a strategy for filling in all of the missing values in the
  dataset. The strategy does not need to be sophisticated.
\end{itemize}

: I used a strategy for filing in all of the missing values with the
mean for that 5-minute interval. First of all, I made a function
\textbf{``interval2steps''} to get the mean steps for particular
5-minute interval.

\begin{Shaded}
\begin{Highlighting}[]
\NormalTok{interval2steps<-}\ControlFlowTok{function}\NormalTok{(interval)\{}
\NormalTok{    stepsInterval[stepsInterval}\OperatorTok{$}\NormalTok{interval}\OperatorTok{==}\NormalTok{interval,]}\OperatorTok{$}\NormalTok{steps}
\NormalTok{\}}
\end{Highlighting}
\end{Shaded}

\begin{itemize}
\tightlist
\item
  Create a new dataset that is equal to the original dataset but with
  the missing data filled in.
\end{itemize}

\begin{Shaded}
\begin{Highlighting}[]
\NormalTok{activityFilled<-activity   }\CommentTok{# Make a new dataset with the original data}
\NormalTok{count=}\DecValTok{0}           \CommentTok{# Count the number of data filled in}
\ControlFlowTok{for}\NormalTok{(i }\ControlFlowTok{in} \DecValTok{1}\OperatorTok{:}\KeywordTok{nrow}\NormalTok{(activityFilled))\{}
    \ControlFlowTok{if}\NormalTok{(}\KeywordTok{is.na}\NormalTok{(activityFilled[i,]}\OperatorTok{$}\NormalTok{steps))\{}
\NormalTok{        activityFilled[i,]}\OperatorTok{$}\NormalTok{steps<-}\KeywordTok{interval2steps}\NormalTok{(activityFilled[i,]}\OperatorTok{$}\NormalTok{interval)}
\NormalTok{        count=count}\OperatorTok{+}\DecValTok{1}
\NormalTok{    \}}
\NormalTok{\}}
\KeywordTok{cat}\NormalTok{(}\StringTok{"Total "}\NormalTok{,count, }\StringTok{"NA values were filled.}\CharTok{\textbackslash{}n\textbackslash{}r}\StringTok{"}\NormalTok{)  }
\end{Highlighting}
\end{Shaded}

\begin{verbatim}
## Total  2304 NA values were filled.
## 
\end{verbatim}

\begin{itemize}
\tightlist
\item
  Make a histogram of the total number of steps taken each day and
  Calculate and report the mean and median total number of steps taken
  per day.
\end{itemize}

\begin{Shaded}
\begin{Highlighting}[]
\NormalTok{totalSteps2<-}\KeywordTok{aggregate}\NormalTok{(steps}\OperatorTok{~}\NormalTok{date,}\DataTypeTok{data=}\NormalTok{activityFilled,sum)}
\KeywordTok{hist}\NormalTok{(totalSteps2}\OperatorTok{$}\NormalTok{steps)}
\end{Highlighting}
\end{Shaded}

\includegraphics{PeerAssessment1_files/figure-latex/unnamed-chunk-9-1.pdf}

\begin{Shaded}
\begin{Highlighting}[]
\KeywordTok{mean}\NormalTok{(totalSteps2}\OperatorTok{$}\NormalTok{steps)}
\end{Highlighting}
\end{Shaded}

\begin{verbatim}
## [1] 10766.19
\end{verbatim}

\begin{Shaded}
\begin{Highlighting}[]
\KeywordTok{median}\NormalTok{(totalSteps2}\OperatorTok{$}\NormalTok{steps)}
\end{Highlighting}
\end{Shaded}

\begin{verbatim}
## [1] 10766.19
\end{verbatim}

\begin{itemize}
\item
  The \textbf{mean} total number of steps taken per day is
  1.0766189\times 10\^{}\{4\} steps.
\item
  The \textbf{median} total number of steps taken per day is
  1.0766189\times 10\^{}\{4\} steps.
\item
  Do these values differ from the estimates from the first part of the
  assignment? What is the impact of imputing missing data on the
  estimates of the total daily number of steps?
\end{itemize}

: The \textbf{mean} value is the \textbf{same} as the value before
imputing missing data because we put the mean value for that particular
5-min interval. The median value shows \textbf{a little} difference :
but it depends on \textbf{where the missing values are}.

\subsection{Are there differences in activity patterns between weekdays
and
weekends?}\label{are-there-differences-in-activity-patterns-between-weekdays-and-weekends}

\begin{itemize}
\tightlist
\item
  Create a new factor variable in the dataset with two levels --
  ``weekday'' and ``weekend'' indicating whether a given date is a
  weekday or weekend day.
\end{itemize}

\begin{Shaded}
\begin{Highlighting}[]
\NormalTok{activityFilled}\OperatorTok{$}\NormalTok{day=}\KeywordTok{ifelse}\NormalTok{(}\KeywordTok{as.POSIXlt}\NormalTok{(}\KeywordTok{as.Date}\NormalTok{(activityFilled}\OperatorTok{$}\NormalTok{date))}\OperatorTok{$}\NormalTok{wday}\OperatorTok\DecValTok{6}\OperatorTok{==}\DecValTok{0}\NormalTok{,}
                          \StringTok{"weekend"}\NormalTok{,}\StringTok{"weekday"}\NormalTok{)}
\CommentTok{# For Sunday and Saturday : weekend, Other days : weekday }
\NormalTok{activityFilled}\OperatorTok{$}\NormalTok{day=}\KeywordTok{factor}\NormalTok{(activityFilled}\OperatorTok{$}\NormalTok{day,}\DataTypeTok{levels=}\KeywordTok{c}\NormalTok{(}\StringTok{"weekday"}\NormalTok{,}\StringTok{"weekend"}\NormalTok{))}
\end{Highlighting}
\end{Shaded}

\begin{itemize}
\tightlist
\item
  Make a panel plot containing a time series plot (i.e.~type = ``l'') of
  the 5-minute interval (x-axis) and the average number of steps taken,
  averaged across all weekday days or weekend days (y-axis). The plot
  should look something like the following, which was creating using
  simulated data:
\end{itemize}

\begin{Shaded}
\begin{Highlighting}[]
\NormalTok{stepsInterval2=}\KeywordTok{aggregate}\NormalTok{(steps}\OperatorTok{~}\NormalTok{interval}\OperatorTok{+}\NormalTok{day,activityFilled,mean)}
\KeywordTok{library}\NormalTok{(lattice)}
\KeywordTok{xyplot}\NormalTok{(steps}\OperatorTok{~}\NormalTok{interval}\OperatorTok{|}\KeywordTok{factor}\NormalTok{(day),}\DataTypeTok{data=}\NormalTok{stepsInterval2,}\DataTypeTok{aspect=}\DecValTok{1}\OperatorTok{/}\DecValTok{2}\NormalTok{,}\DataTypeTok{type=}\StringTok{"l"}\NormalTok{)}
\end{Highlighting}
\end{Shaded}

\includegraphics{PeerAssessment1_files/figure-latex/unnamed-chunk-11-1.pdf}

\subsection{Imputing missing values}\label{imputing-missing-values}

\begin{itemize}
\tightlist
\item
  Calculate and report the total number of missing values in the dataset
  (i.e.~the total number of rows with NAs)
\end{itemize}

\begin{Shaded}
\begin{Highlighting}[]
\KeywordTok{sum}\NormalTok{(}\KeywordTok{is.na}\NormalTok{(activity}\OperatorTok{$}\NormalTok{steps))}
\end{Highlighting}
\end{Shaded}

\begin{verbatim}
## [1] 2304
\end{verbatim}


\end{document}
